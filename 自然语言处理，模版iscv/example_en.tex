\documentclass[10pt]{article}
\usepackage[ENG]{iscv}
\usepackage{capt-of}
\usepackage{hologo}
\bibliographystyle{ieeetr}

\title{Template for ISCV}
\author{First Author \and Second Author}
\setcounter{page}{1}

\begin{document}

% \twocolumn[{%
% \maketitle
% \begin{center}
%     \centering
%     \includegraphics[width=.9\linewidth,height=4cm]{example-image-golden}
%     \captionof{figure}{Teaser Image}
% \end{center}%
% }]
\maketitle

\abstract
This is where the abstract should be placed. It should consist
of one paragraph and a concise summary of the material
discussed in the article below. It is preferable not to use
footnotes in the abstract, the title or indeed anywhere in the paper. 
The acknowledgement for funding organisations etc. is placed in 
a separate section at the end of the text. We wish you success 
with the preparation of your manuscript.

\keywords
Maximum 5 keywords placed before the abstract.

\section{Introduction}
With the ISCV package file, a computer running \LaTeX{} and a basic understanding of the \LaTeX{} language,
an author should be able to produce professional quality typeset research papers with minimal effort.
The purpose of this article is to serve as a user guide of the ISCV \LaTeX{} package file including
the basic typesetting rules for ISCV journal.

\section{Basic Usage}

To start an article for ISCV, your tex file should begin with:
\begin{verbatim}
\documentclass[10pt]{article}
\usepackage[ENG]{iscv}
\end{verbatim}
If your paper is mostly written in Chinese, you should use:
\begin{verbatim}
\usepackage[CHN]{iscv}
\end{verbatim}
You should use \hologo{XeLaTeX} to compile your paper to enable Chinese fonts. If you use Overleaf, simply set compiler to XeLaTeX in the project menu.

\section{Manuscript preparation}
Papers submitted for ISCV will be reviewed by the ISCV reviewer committee. All papers should be typed in either English or Chinese Mandarin.
This instruction page is an example of the format and font sizes to be used. \LaTeX{} is preferred as an official template is available. 
MS word users need to help themselves. 

Three types of submissions can be accepted in ISCV:
\begin{itemize}
    \item Full Paper: At most 4 pages
    \item Tech Report: At most 2 pages
    \item Poster and Newsletter: 1 page
\end{itemize}
Figures and tables are included and references
are excluded in page counts.  The authors may optionally submit supplementary materials, including videos, additional 
figures or detailed analysis on the forum. Authors of accepted papers will be invited for presentations at the symposium. 

All types of papers should fall within a frame of {14 cm $\times$ 20 cm} centred on a B5 page (17.6 cm $\times$ 24 cm). 
Specifically, full papers and tech reports should be typed in two-columns. 

\subsection{Figures and tables}
Figures and tables should be centred in the column, numbered
consecutively throughout the text, and each should have a
caption underneath it (see for example Table~\ref{tab:example}). 
Care should be taken that the lettering is not too small. 
You can use \texttt{figure*} or \texttt{table*} to create two-column figures 
or tables, which should be put on the top or bottom of each page.

\begin{figure}[h]
\centering
\includegraphics[width=.5\linewidth]{figs/fig1.pdf}
\caption{This is an example of a figure caption.} 
\label{fig:example}
\end{figure}

\begin{table}[h]
\centering\begin{tabular}{ccc}
\toprule
Net & top1 & top5 \\
\midrule
ResNet & 7\% & 5\% \\
VGG & 10\% & 7\% \\
ShuffleNet & 15\% & 10\% \\
\bottomrule
\end{tabular}
\caption{This is an example of a table caption.}
\label{tab:example}
\end{table}

\subsection{Equations}
Equations should be typed within the text, centred, and should
be numbered consecutively throughout the text. They should
be referred to in the text as Equation (n). Their numbers
should be typed in parentheses, flush right, as in the following
example. You can use \verb|\eqref{eq:example}| to refer an equation, such as Eq.~\eqref{eq:example}.
\begin{equation}
\label{eq:example}
	    PA + A'P - PBR^{-1}B'P + Q  =  0 \enspace.
\end{equation}

\section{Generating a {PDF} file}
The PDF format will be the final format under which the
papers will appear in the Proceedings. Therefore you are
required to submit your paper as a PDF document. 

Overleaf is recommended to process \LaTeX{} files.

\section{Your References}
The list of references should be ordered in the same order as
first cited in the text. All references should be cited in the
text, and using square brackets such as~\cite{ref01} and~\cite{ref01,ref02}. 

\acknowledgements
The acknowledgement for funding organisations etc. should
be placed in a separate section at the end of the text.

Thank you for your cooperation in complying with these
instructions.

\bibliography{iscv}


\end{document}
